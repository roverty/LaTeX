\documentclass{article} %O report o book

%Paquetes vistos en la clase pasada
\usepackage[utf8]{inputenc}
\usepackage[spanish]{babel}

%Paquetes que nos permitirán la inserción de ecuaciones y símbolos
%matemáticos en LaTeX

\usepackage{amsmath}
\usepackage{amssymb}
\usepackage{amsfonts}
\usepackage{latexsym,cancel}
\usepackage{esvect}
\usepackage{palatino,eulervm} 

%https://github.com/roverty/LaTeX/tree/main/03-matematica-ciencia

%Nuevos comandos

%\newcommand{\sen}{\mathop{\rm sen}\nolimits} %seno
%\newcommand{\arcsen}{\mathop{\rm arcsen}\nolimits}
%\newcommand{\arcsec}{\mathop{\rm arcsec}\nolimits}
\newcommand{\R}{\mathbb{R}}
\newcommand{\N}{\mathbb{N}}
\newcommand{\Z}{\mathbb{Z}}
\renewcommand{\thepage}{}

\parindent=0mm%%%
\parskip=3mm

\begin{document}
  
%\noindent
\textbf{Ecuaci\'on sencilla en \LaTeX}
Esta es la forma más sencilla de escribir una ecuaci\'on en \LaTeX sin la
necesidad de escribir los paquetes en el pre\'ambulo del documento, ya que
\LaTeX carga este tipo de ecuaciones por defecto
con los signos de pesos \$...\$ 

La fórmula $ f(x) = y $

Sin embargo, cuando queremos implementar matrices, integrales y s\'imbolos
matem\'aticos m\'as complejos tendremos que recurrir a los entornos que los
paquetes matem\'aticos que \LaTeX ofrece.
Dicho paquetes son: amsmath, amssymb, amsfonts y latexsym.

\section*{1. Potencias, sub\'indices y super\'indices}
Exponente: $x^p$

Potencia de potencia: $(2^2)^n$

Sub\'indice: $a_n$

Sub\'indice compuesto: $u_{N + 1}$ $a_{n+1}$ $u_{_{N+1}}$ $u_{ij}$

Super\'indice y sub\'indice a la vez: $a_i^j$

Sumatoria: $\sum_{n = 1}^{N}u_n$

Super\'indice compuesto: $x^{n+1}$

Potencia elevada a otra: $2^{2^n}$

Integral: $\int_a^b f(x)\ dx$


\section*{2. Tamaño natural}

El comando displaystyle se ocupa para desplegar en tamaño 
natural, para que éste no s\'olo se ajuste al ancho del rengl\'on \\
La suma parcial $N-$ésima $S_N$ se define con la igualdad $S_N=\sum_{k=1}^{N} \; a_n$
La suma parcial $N-$ésima $S_N$ se define con la igualdad $\displaystyle
S_N=\sum_{k=1}^{N} \; a_n$


\section*{3. Fracciones y expresiones de dos niveles}
\begin{enumerate}
\setlength\itemsep{1cm}%%%
\item ${x+1 \over x-1}$

\item $\frac{x+1}{x-1}$

\item $\dfrac{x+1}{x-1}$

\item $\tfrac{x+1}{x-1}$

\item ${{x+1 \over 3} \over x-1}$

\item $\displaystyle{\left( 1+ {1 \over x} \right)^{n+1 \over n}}$

\item $\displaystyle \left( 1+ \frac{1}{x} \right)^\frac{n+1}{n}$

\item $\displaystyle{\left( 1+ {1 \over x} \right)}^{\displaystyle{n+1 \over n}}$

\item ${x+1 \atop x-1}$

\item ${x+1 \above 2pt x-1}$

\item ${x+1 \brace x-1}$

\item ${x+1 \brack x-1}$

\item $\displaystyle{a \stackrel{f}{\rightarrow} b}$

\item $\displaystyle{\lim_{ x \rightarrow 0}} f(x)$

\item $\displaystyle{a \choose b}$

\item $\displaystyle{\sum_{\substack{0<i< m\\0<j<n}}a_ib_j}$
%Ejercicio
\item $$ L_{n,k}(x)
= \prod_{\overset{i=0}{i\neq k}}^{n}\,\frac{x-x_i}{x_k-x_i}
= \frac{(x-x_0)(x-x_1)\cdots(x-x_{k-1})(x-x_{k+1})\cdots(x-x_n)}{
(x_k-x_0)\cdots (x_k-x_{k-1})(x_k-x_{k+1})\cdots(x_k-x_n)
}  $$
\end{enumerate}

\section*{4.Integrales}
\begin{enumerate}
\setlength\itemsep{1cm}
\item Integral de l\'inea $\displaystyle{\int_C\boldsymbol{F}\cdot\, dr}$

\item Integral cerrada $ \displaystyle{\oint_C\pmb{F}\cdot\, dr} $

\item Integral doble $\displaystyle{{\iint_D f(x,y)\,dA}}$

\item Integral triple $\displaystyle{{\iiint_Q f(x,y,z)\,dA}}$
\end{enumerate}
\section*{5. Ra\'ices}
\begin{enumerate}
\setlength\itemsep{1cm}
\item 1. Ra\'iz cuadrada: $\sqrt{x+1}$

\item 2. Ra\'iz compuesta: $ \displaystyle{ \sqrt[n]{x+\sqrt{x}} }$

\item 3. Ra\'iz con diferente indice: $\sqrt[3]{x}$

\item 4. Ra\'z con un conciente dentro: $\sqrt{\frac{x+y}{x-y}}$
\end{enumerate}

\section*{7. Delimitadores}
$\displaystyle \left[{x+1 \over (x-1)^2} \right]^n$
\newline
Integral con límites: $\int_{a}^{b} \left(x^2+\sin (x)\right)\, dx = \left.
\dfrac{x^3}{3}-\cos (x)\right|_{a}^{b}$

\section*{Funci\'on a trozos}
\[f(x)=\left\{ \begin{array}{rcl}
    x^2+1 & \mbox{si} & x\geq 0\\
    & & \\
    \ln|x| & \mbox{si} & x< 0\\
    \end{array}
\right. \] % Observe el punto que cierra: \left\{ ... \right.
    
\section*{8. Llaves y barras horizontales}
Leyes de DeMorgan:
$\displaystyle{ \left\{{ \overline{A \cup B} = \overline{A} \cap
\overline{B}\atop\overline{A \cap B} = \overline{A}\cup
\overline{B}}\right.} $ 

\section*{9. Acentos y "sombreros"}
Vector unitario: $\hat{\imath}$ \\
Vector: $\bar{p}$ \\
Acento: $\acute{a}$ \\
Con flecha: $\vec{p}$ \\

\section*{10 Vectores}
\begin{enumerate}
\setlength\itemsep{1cm}
\item Notación normal de un vector: $\vv{v}$
\item Vector en may\'uscula: $\vv{A}$
\item Producto cruz: $\vv{v \times w}$
\end{enumerate}


\section*{11. Negritas}

$\pmb{\cos(x+2\pi)=\cos x}$ \\
$\cos(x+\pmb{2\pi})=\cos x$



\section*{12. Espaciado}
$\int f(x)dx$ \\
$\displaystyle{\int} f(x)\, dx$
\section*{13. Centrado}
\[ ab \leq \left( {a+b \over 2} \right)^2 \]
\section*{Entorno begin equation}
\begin{equation} \label{eu_eqn}
    e^{\pi i} - 1 = 0
\end{equation}
\section*{14. Arreglos}
\[
A = \left( \begin{array}{lcr}
a & a+b & k-a \\
b & b & k-a-b \\
\vdots & \vdots & \vdots \\
z & z + z & k-z
\end{array}
\right)
\]

Otra funci\'on a trozos
\[
f(x)= \left\{ \begin{array}{lcl}
x^2 & \mbox{ si } & x<0 \\
& & \\
x-1 & \mbox{ si } & x>0
\end{array}
\right.
\]

\section*{15. Matrices}

Matriz simple
$\bigl( \begin{smallmatrix}
    a & b \\ c & d
    \end{smallmatrix} \bigr)$.

Matriz entre corchetes

\begin{equation*}
    \begin{bmatrix}
    6 & 8 & 1\\
    2 & 9 & 3\\
    4 & 5 & 1
    \end{bmatrix}
\end{equation*}

Matriz entre paréntesis

\begin{equation*}
    \begin{pmatrix}
    2 & 5 & 0\\
    7 & 3 & 8\\
    3 & 0 & 1
    \end{pmatrix}
\end{equation*}

Matriz entre barras

\begin{equation*}
    \begin{vmatrix}
    2 & 5 & 8\\
    6 & 7 & 1\\
    5 & 0 & 3
    \end{vmatrix}
\end{equation*}

Matriz entre barras dobles
\begin{equation*}
    \begin{Vmatrix}
    3 & 5 & 0\\
    2 & 8 & 6\\
    7 & 1 & 4
    \end{Vmatrix}
\end{equation*}

Matriz entre Llaves
\begin{equation*}
    \begin{Bmatrix}
    1 & 7 & 8\\
    0 & 5 & 7\\
    9 & 3 & 4
    \end{Bmatrix}
\end{equation*}
Matriz con puntos
\begin{equation*}
    \begin{pmatrix}
    1 & 0 & \cdots & 0\\
    0 & 1 & \cdots & 0\\
    \vdots & \vdots & \ddots & \vdots\\
    0 & 0 & \cdots & 1
    \end{pmatrix}
\end{equation*}
    

\section*{16. Alineamiento}
Sin numeración 
\begin{eqnarray*} % Espacio entre filas aumentado \\[0.2cm]
    \mbox{mcd}(a,b) & = & \mbox{mcd}(a-r_0q,r_0) \\[0.2cm]
    & = & \mbox{mcd}(r_1,r_0) \\[0.2cm]
    & = & \mbox{mcd}(r_1,r_0-r_1q_2)\\[0.2cm]
    & = & \mbox{mcd}(r_1,r_2) \\[0.2cm]
    & = & \mbox{mcd}(r_1-r_2q_2,r_2)\\[0.2cm]
\end{eqnarray*}
Con numeración 

\begin{eqnarray}
    y=\sqrt[n]{x} &\Longrightarrow &y^n=x \\
    &\Longrightarrow &n\log\,y=\log\,x;\;\mbox{si}\; x,y>0\\
    &\Longrightarrow &\log \sqrt[n]{x}={1 \over n}\log \,x
\end{eqnarray}

Numeración selectiva
\begin{eqnarray}
    y=\sqrt[n]{x} &\Longrightarrow &y^n = x \nonumber\\[0.5cm]
    &\Longrightarrow &n\log \,y=\log\,x,
    \;\mbox{si}\; x,y>0 \\[0.5cm]
    &\Longrightarrow &\log \sqrt[n]{x}={1 \over n}\log \,x
\end{eqnarray}

Align

\begin{align*}
    \intertext{Agrupamos,}
    \frac{a+ay+ax+y}{x+y} &= \frac{ax+ay+x+y}{x+y} &\mbox{Agrupar}\\
    \intertext{sacamos el factor común,}
    &= \frac{a(x+y)+x+y}{x+y} &\mbox{Factor común}\\
    &= \frac{(x+y)(a+1)}{x+y} &\mbox{Simplificar}\\
    &= a+1
\end{align*}

\section{Nuevos comandos}

Nuevo seno
$\sen$ \\
Nuevo angulo cuyo seno
$\arcsen$ \\ 
Reales
$\R$ \\
Naturales
$\N$ \\
Enteros
$\Z$ \\

\end{document}
