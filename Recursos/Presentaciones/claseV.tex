\documentclass{beamer}

\usetheme{Luebeck}
\usecolortheme{orchid}

\usepackage[spanish]{babel}
\usepackage[utf8]{inputenc}
\usepackage{lipsum}
\usepackage{graphicx}

\author{Armando Rivera}
\date{\today}
\title{Mi primera presentación con \LaTeX}
\subtitle{Introducción}
\institute[UNAM]{Universidad Nacional Autónoma de México\\Facultad de Ingenieía}


\begin{document}
	\begin{frame}
		\maketitle	
	\end{frame}
	\begin{frame}{Índice}
	%\tableofcontents[pausesections]
	\tableofcontents
	\end{frame}
	\section{Introducción}
	
	\begin{frame}{Introducción}{Sub título}
		Aquí hay texto que puedo colocar en mi diapositiva
	\end{frame}

	%\begin{frame}[plain]{Introducción}{Segundo frame}
	\begin{frame}{Introducción}{Segundo frame}
		Aquí hay otro frame con el que probaremos algunas opciones
	\end{frame}
\section{Bloques}
	\begin{frame}{Bloques}{block}
	Aquí tenemos un ejemplo de un bloque
	\begin{block}{Pitágoras}
			$ a^2+b^2=c^2 $
	\end{block}		
	\end{frame}
	\begin{frame}{Bloques}{Alertas}
	Aquí hay una alerta
	\begin{alertblock}{CUIDADO}
		$ (a+b)^2\neq a^2+b^2 $
	\end{alertblock}
	\end{frame}
	\begin{frame}{Bloques}{Ejemplos}
		\begin{exampleblock}{Ejemplo}
			Aquí tenemos un ejemplo
		\end{exampleblock}
	\end{frame}
	\section{Diapositivas dinámicas}
	\begin{frame}{Diapositivas dinámicas}
		\transsplitverticalout
		Este frame es un ejemplo de la transiciones que podemos realizar al cambiar de diapositiva\pause
		También podemos realizar pausas
	\end{frame}
	{
			
			\usebackgroundtemplate{
				\includegraphics[width = \paperwidth, height=\paperheight]{img/fondo}
			}
			\begin{frame}{Fondo de diapositivas}
			Esta diapositiva tiene el fondo cambiado
		\end{frame}
	}
	
\end{document}