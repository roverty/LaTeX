\documentclass{report}
\usepackage[utf8]{inputenc}
\usepackage[spanish]{babel}
\usepackage{lipsum}
\usepackage{amssymb}
\usepackage{amsmath}
\usepackage{xcolor}
\usepackage{xstring}
\usepackage{caption}
\usepackage{cleveref}
\usepackage{soul}
\usepackage{ulem}
\usepackage{algorithmic}
\usepackage{algorithm}

\title{Clase de Figuras}
\author{Armando Rivera}
\date{\today}
\definecolor{myGreen}{HTML}{36A736}	%Este codigo es del color verde
\definecolor{myBlue}{HTML}{02528F}
\definecolor{myOrange}{HTML}{FF4312}
\definecolor{blue254}{HTML}{02528F}
\floatname{algorithm}{Algorítmo}

\DeclareCaptionType{Ecuacion}

\newcommand{\Fourier}[1][eq] {
	\IfEqCase {#1} {	%Indoca que va a trabajar con el primer parametro
		{eq} {
			\begin{equation}
			\hat{f}(\xi)=\int_{-\infty}^{\infty}f(x)e^{-2\pi ix \xi}dx
			\end{equation}
		}
		{disp} {
			\[
			\hat{f}(\xi)=\int_{-\infty}^{\infty}f(x)e^{-2\pi\xi}dx
			\]
		}
		{inline} {	%Le estamos diciento que va a colorear la ecuacion segun mandemos el color en el segundo parametro
			$\hat{f}(\xi)=\int_{-\infty}^{\infty}f(x)e^{-2\pi\xi}dx$}
	}
}
\newcommand{\Vector}{(x_1,x_2,\dots,x_n)}
% $ \vector $
\newcommand{\vectorr}[1]{(#1_1,#1_2,\dots,#1_n)}
% $ \vector{a} $
\newcommand{\miVector}[2]{(#1_1,#1_2,\dots,#1_#2)}
% $ \miVector{x}{m} $
\newcommand{\nuevovector}[2][x]{(#1_1,#1_2,\dots,#1_#2)}
% El comando \nuevovector, recibe dos parámetros, uno opcional y el otro obligatorio, por defecto toma la x
% $ \nuevovector{n}    \nuevovector[a]{n} $

\newtheorem{theorem}{Definicion}
\newtheorem{demo}{Demostracion}
\newtheorem{Ejem}{Ejemplo}[section]


\begin{document}
	\maketitle
	\st{Hola mundo}\\
	\uline{Hola mundo}\\
	\uuline{Hola mundo}\\
	\uwave{Hola mundo}\\
	\xout{Hola mundo}\\
	\st{Hola mundo}\\
	\ul{Hola mundo}
	

	\lipsum[1]
	
	
	\chapter{Ambiente Matemático}
	Para poder trabajar con símbolos matemáticos necesitamos usar AMS-LaTeX, ésta es una colección de paquetes que nos serán de ayuda a la hora de escribir ecuaciones matemáticas. Pertenece a American Mathematical Society.\\ Debemos usar el paquete \textbf{amsmath} para escribir ecuaciones, no olvidar incluilo al escribir ecuaciones. 
	Para incluir ecuaciones necesitamos usar el símbolo de \$ ejemplo: \par
	Si $a > b$ y $b>c$ $\therefore$ $a>c$ \ldots\\\\
	$a_n=a_n-1+a_n-2$\\	%Error
	$a_n=a_{n-1}+a_{n-2}$\\	%Correcto
	
	$f(x)=e^ax+b-c$			\\%Error
	$f(x)=e^{ax+b}-c$		\\	%Correcto
	
	$P(z)=\alpha + \lambda(z-\alpha)^n$\\
	Existen comandos para las letras griegas y la notación común\\
	$\iint_Sdxdy=\iint_{H(s)}dxdy$\\
	\\\\Usando display\\
	Una ecuacion usando el comando display $\displaystyle{\int_a^bf(x)dx}$
\\Otro ejemplo de como mostrar una ecuacion es la siguiente:
\[
p(x)+\frac{Q(x)}{P(x)}, q(x)=\frac{R(x)}{P(x)}
\]	De esta forma la ecuacion se pone en una nueva linea, se centra y no se marca el numero de la ecuacion

\begin{gather}
	\sqrt[5]{x}\\
	\prod_{i=0}^{n}\\
	\sqrt{x}\Leftrightarrow x^{1/2}
\end{gather}



\section{Ambiente equation}
Las raíces de la ecuación cuadrática están dadas por
\begin{Ecuacion}

\begin{equation}
x=\frac{-b\pm \sqrt{b^2-4ac}}{2a}
\label{eq:Formula general}
\end{equation}
\caption{Formula general}
\end{Ecuacion}
donde $a$, $b$ y $c$ son \ldots

%La presentacion de ecuaciones varia entre el modo inline y el modo display (ambiente equation)
Escribir $\Omega=\sum_{k=1}^n \omega_k$ es diferente a escribir: 
\begin{equation*}
\Omega=\sum_{k=1}^n\omega_k
\end{equation*}

La formula general se ubica en la ecuación \cref{eq:Formula general}
\begin{equation}
\lim_{x\rightarrow y}{(1-x)^2+100(y-x^2)^2}
\end{equation}

\begin{align*}
E(s)&=R(s)-B(s)\\
Y(s)&=E(s)G(s)\\
G(s)&=\frac{Y(s)}{E(s)}\\
\end{align*}
\newcommand{\cuadratica} {
	\begin{equation}
	x_{1,2}=\frac{1}{2a}\Bigl(-b\pm \sqrt{b^2-4ac}\Bigr)
	\end{equation}
}


Aqui hay una ecuacion cuadratica \cuadratica


\lipsum[1]
\Fourier[disp]\\
\lipsum[1]

\subsection{Tipos de acentos}

En \LaTeX podemos ocupar varios tipos de acentos usados en artículos científicos\\
\begin{gather}
a'\qquad a''\qquad a'''\qquad a''''\qquad\\
\hat{a}\qquad\bar{a}\qquad\overline{ABC}\qquad\check{a}\qquad\tilde{a}\\
\grave{a}\qquad\acute{a}\qquad\breve{a}\qquad\vec{a}\\
\dot{a}\qquad\ddot{a}\qquad\dddot{a}\qquad\ddddot{a}\\
\not{a}\qquad\mathring{a}\qquad\widehat{ABC}\qquad\widetilde{ABC}
\end{gather}




\section{Signos de agrupacion}
$
\{a+b\}\\
\langle a+b	\rangle\\
\lvert a+b	\rvert \\
\lVert a+b	\rVert \\
\lfloor	a+b \rfloor\\
\lceil	a+b \rceil\\$
[a + b]

\[
h_\theta(x)=g\Bigg(\frac{1}{1+e^{(-\theta^Tx)}}\Bigg)
\]


%\left y \right no es suficiente para ajustar los signos de agrupacion a su contenido, puede utilizarse entonces \bigl, \bigr, \Bigl, \biggl, \biggr, \Biggl, \Biggr

\[
\frac{d}{dx}\Biggl(\frac{dy}{dx}\Biggr)
\]


\section{Matrices}

\[
W(f_1, f_2)(x)=
\begin{vmatrix}
x^2 & x|x|\\2x & \frac{2x^2}{|x|}
\end{vmatrix}
\]

El ambiente array permite alinear por separado cada columna de una matriz
\[\left[
\begin{array}{lcr}	%Aqui nos dice que la primera columna va a estar alineada a la izq, la segunda en el centro y la tercera a la derecha
-0.1 & a & 0.1 \\ -0.01 & a+1 & 0.01
\end{array} \right]
\]

El comando split permite escribir ecuaciones de varias lineas alineadas por un caracter
\[
\begin{split}
L[c_1y_1 + c_2y_2] &= c_2L[y_1'' + py_1']\cdots \\
&= C_2L[y_1] + c_2L[y_2]
\end{split}	%En este caso las lineas estan alineadas por el signo igual, con este ambiente no se enumeran las ecuaciones
\]

También podemos hacer uso del ambiente \texttt{gather} Este ambiente permite agrupar un conjunto de ecuaciones numeradas sin carácter de alineación, si quisiéramos que alguna ecuación no este numerada agregamos el comando notag al final de la ecuación
\begin{gather}
W' + p(x)W=0 \\
W(x) = Ce^{-\int p(x) dx}	\notag 	%Esta ecuacion no se mostrara numerada
\end{gather}


\begin{equation*}
\mathbf{X} = \left(
\begin{array}{ccc}
x_1 & x_2 & \ldots \\
x_3 & x_4 & \ldots \\
\vdots & \vdots & \ddots
\end{array} \right)
\end{equation*}


$\mathbb{R}$

El paquete amsmath nos provee de ambientes para crear matrices de una forma sencilla, se tienen 6 matrices disponibles y un ambiente cases.
\begin{itemize}
	\item cases
	\item matrix sin delimitador
	\item pmatrix
	\item bmatrix
	\item Bmatrix
	\item vmatrix
	\item Vmatrix
\end{itemize}

\begin{equation*}
|x| =
\begin{cases}
-x & \text{if } x < 0,\\
0 & \text{if } x = 0,\\
x & \text{if } x > 0.
\end{cases}
\end{equation*}


\begin{equation*}
\begin{matrix}
1 & 2 \\
3 & 4
\end{matrix}
\qquad
\begin{bmatrix}
a_{11} & a_{12} & a_{13}\\
a_{21} & a_{22} & a_{23}\\
a_{31} & a_{32} & a_{33}
\end{bmatrix}
\end{equation*}
\[
\begin{Bmatrix}
	a_{11} & a_{12} & a_{13}\\
	a_{21} & a_{22} & a_{23}\\
	a_{31} & a_{32} & a_{33}
\end{Bmatrix}
\]

\[
\begin{pmatrix}
a_{11} & a_{12} & a_{13}\\
a_{21} & a_{22} & a_{23}\\
a_{31} & a_{32} & a_{33}
\end{pmatrix}
\]

\[
\begin{vmatrix}
a_{11} & a_{12} & a_{13}\\
a_{21} & a_{22} & a_{23}\\
a_{31} & a_{32} & a_{33}
\end{vmatrix}
\]
\[
\begin{Vmatrix}
a_{11} & a_{12} & a_{13}\\
a_{21} & a_{22} & a_{23}\\
a_{31} & a_{32} & a_{33}
\end{Vmatrix}
\]

Ejemplo de Serie de Fourier \Fourier[eq]


\begin{theorem}[Ejemplo de definición]
	Este es un ejemplo del ambiente \texttt{theorem}
\begin{gather}
W' + p(x)W=0 \\
W(x) = Ce^{-\int p(x) dx}	\notag 	%Esta ecuacion no se mostrara numerada
\end{gather}
\[
W(f_1, f_2)(x)=
\begin{vmatrix}
x^2 & x|x|\\2x & \frac{2x^2}{|x|}
\end{vmatrix}
\]
\end{theorem}

\begin{demo}%[Ejemplo de Demostración]
	Esto es una demostración
	\label{dem:Ejemplo}
\end{demo}
El ejemplo de demostracion \ref{dem:Ejemplo} es un ejemplo de una demostracion usando el ambiente thorem

\chapter{Ejemplos}
\lipsum[2-2]
\section{Esta es una sección}
$\heartsuit\clubsuit$

\lipsum[2-2]
\begin{Ejem}
	Considerese \dots \lipsum[2-2]
	\begin{equation}
	G(s)=\begin{array}{cc}
	[1& 0]
	\end{array} 
	\Biggl\{
	\left[
	\begin{array}{cc}
		s&-1\\\frac{K}{m}&s+\frac{b}{m}
	\end{array}
	\right]
	\Biggr\}^{-1}
	\left[
	\begin{array}{c}
		0\\\frac{1}{m}
	\end{array}
	\right]
\end{equation}

	\end{Ejem}


\chapter{Incluir código}
Para incluir código tenemos que usar el paquete \textit{algorithmic}y del \textit{algorithm} y debemos de iniciar el entorno \textit{algorithmic} y dentro de él encerrar el presudocódigo.


\begin{algorithm}
	\caption{Ejemplo de pseudocódigo}
	\begin{algorithmic}[1]
		\REQUIRE Aquí incluimos los requerimientos del programa
		%\ENSURE $y = x^n$
		%\STATE $y \leftarrow 1$
		\IF{Condición}
		\STATE Ejecuta esta linea
		\STATE También ejecuta esta
		\ELSE
		\STATE De lo contrario ejecuta esta linea
		\STATE Y después esta
		\ENDIF
		\WHILE{Condición}
		\IF{Condición de if}
		\STATE Más lineas de código a ejecutar
		\STATE Otra línea más
		\ELSE[Podemos agregar condiciones a un else]
		\STATE Y después más líneas de código
		\STATE Esta es la última
		\ENDIF
		\ENDWHILE
		\STATE Código \COMMENT{Aquí hay texto comentado}
		\LOOP
		\STATE this processing will be repeated forever
		\ENDLOOP
	\end{algorithmic}
\end{algorithm}

\chapter{Tarea}

$( a^2+b^2)=c^2\\\\ c=\sqrt{a^2+b^2}\\\\
A\oplus B=A\overline{B}+\overline{A}B \\\\
\text{El valor de: } R_1 \text{ es: } 300[\Omega] \\\\
Z(X)=\frac{X-\mu}{\sigma}\\\\
\sum_{i=0}^nn=\frac{n(n+1)}{2}\\\\
\sin^2{x}+\cos^2{x}=1\\\\
F(\omega)=\int^\infty_{-\infty}{f(t)e^{-j\omega t}\delta t}$


\end{document}