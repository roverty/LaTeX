\documentclass{IEEEtran}

\usepackage[spanish]{babel}
\usepackage[utf8]{inputenc}
\usepackage{soul}
\usepackage{ulem}
\usepackage{lipsum}
\usepackage{vmargin}
\spanishdecimal{.}


\newtheorem{Corol}{Corolario}
\newtheorem{Ejem}{Ejemplo}[section]

\begin{document}
	
	\renewcommand{\contentsname}{Mi Indice}
	\tableofcontents
	\renewcommand{\listtablename}{Índice de Tablas} 
	\renewcommand{\tablename}{Tabla}
	\newpage
	\listoftables
	\newpage
	\section*{Introducción}
	$ 5.5 $
	
	\newcommand{\cuadratica}{
		\begin{equation}
		x=\frac{-b\pm \sqrt{b^2-4ac}}{2a}
		\end{equation}
	}
	
	\begin{Corol}
		Esto es un corolario \lipsum[1]
		\cuadratica
	\end{Corol}
	
	
	
	\lipsum[1-10]
	\begin{table}[h]
		\centering
		\begin{tabular}{|c|c|}
			\hline
			\multicolumn{2}{|c|}{Primera Medición}
			\\\hline
			Nivel(cm)&Ángulo$\theta$\\\hline
			0&28\\\hline
			1&30\\\hline
			2&33\\\hline
			3&35\\\hline
			4&37\\\hline
			5&38.5\\\hline
			6&41\\\hline
			7&44\\\hline
		\end{tabular}
		\hspace{1cm}
		\begin{tabular}{|c|c|}
			\hline
			\multicolumn{2}{|c|}{Segunda Medición}
			\\\hline
			Nivel(cm)&Ángulo$\theta$\\\hline
			0&25\\\hline
			1&29\\\hline
			2&32\\\hline
			3&37\\\hline
			4&38\\\hline
			5&39\\\hline
			6&40.5\\\hline
			7&43\\\hline
		\end{tabular}
		\caption{Datos obtenidos con el prototipo de nivel}
		\label{Tab:DPN}
	\end{table}
	
	\section{Cursos de proteco}
		\section{Python}
		\section{Básico}	
			
		\begin{Ejem}
			Esto es un ejemplo de la fórmula general para resolver ecuaciones de segundo grado
			\cuadratica
		\end{Ejem}	
			\begin{Ejem}
				Esto es un ejemplo de la fórmula general para resolver ecuaciones de segundo grado
				\cuadratica
			\end{Ejem}
			
			\newcommand{\Vector}{x_1, x_2, \dots, x_n}
			\newcommand{\vectorr}[1]{#1_1, #1_2,\dots, #1_n}
			\newcommand{\miVector}[2]{#1_1, #1_2,\dots, #1_#2}
			\newcommand{\unVector}[2][x]{#1_1, #1_2,\dots, #1_#2}
			
			\cuadratica
			
			$ \vectorr{y} $\\
			$ \vectorr{h} $\\			
			$ \Vector $\\
			$ \miVector{p}{q} $
			
			
			
			\lipsum[1-2]
			
			\cuadratica
			
			\lipsum[1-10]
			\subsection{Variables}	\lipsum[1-10]
			\subsection{Secuencias de control}	\lipsum[1-10]
				\subsubsection{Ciclos}	\lipsum[1-10]
				\subsubsection{Condicionales}	\lipsum[1-10]
			\section{Intermedio}
				\subsection{Variables}	\lipsum[1-10]
				\subsection{Secuencias de control}	
					\subsubsection{Ciclos}	
					\subsubsection{Condicionales}	\lipsum[1-10]
		\section{\textit{LaTeX}}	\lipsum[1-10]
\section{Básico}	\lipsum[1-10]
\subsection{Variables}	\lipsum[1-10]
\subsection{Secuencias de control}	
\subsubsection{Ciclos}	\lipsum[1-10]
\subsubsection{Condicionales}
\section{Intermedio}	
\subsection{Variables}	\lipsum[1-10]
\subsection{Secuencias de control}	\lipsum[1-10]
\subsubsection{Ciclos}	\lipsum[1-10]
\subsubsection{Condicionales}	\lipsum[1-10]

			
				
		\section{\textit{Java}}
		Como podemos ver en la tabla \ref{Tab:E-M}
					\lipsum[1-10]
			\section{Básico}	\lipsum[1-10]
				\subsection{Variables}	\lipsum[1-10]
				\subsection{Secuencias de control}	\lipsum[1-10]
					\subsubsection{Ciclos}	\lipsum[1-10]
					\subsubsection{Condicionales}
				\section{Intermedio}	\lipsum[1-10]
					\subsection{Variables}	\lipsum[1-10]
					\subsection{Secuencias de control}	\lipsum[1-10]
						\subsubsection{Ciclos}	\lipsum[1-10]
						\subsubsection{Condicionales}	\lipsum[1-10]
					
\end{document}