\documentclass{article}
\usepackage{amsmath} % Requerido para elementos matemáticos
\usepackage[utf8]{inputenc}			
\usepackage[spanish]{babel} %Ajustar idioma: español
\spanishdecimal{.}
\usepackage{graphicx} % Requerido para la inclusión de imágenes
%\usepackage{natbib} % Requerido para el cambio de formato a formato APA
\usepackage{lipsum}
\usepackage{vmargin}


\begin{document}
	\renewcommand{\tablename}{Tabla}
	\begin{equation}
		F(\omega)=\int^\infty_{-\infty}{f(t)e^{-j\omega t}\delta t}
		\label{eq:Ecuacion}
	\end{equation}
	
	Cómo podemos ver en la ecuación \eqref{eq:Ecuacion} 
	
	\begin{table}[h]
		\centering
		\begin{tabular}{|c|c|}
			\hline
			\multicolumn{2}{|c|}{Primera Medición}
			\\\hline
			Nivel(cm)&Ángulo$\theta$\\\hline
			0&28\\\hline
			1&30\\\hline
			2&33\\\hline
			3&35\\\hline
			4&37\\\hline
			5&38.5\\\hline
			6&41\\\hline
			7&44\\\hline
		\end{tabular}
		\hspace{1cm}
		\begin{tabular}{|c|c|}
			\hline
			\multicolumn{2}{|c|}{Segunda Medición}
			\\\hline
			Nivel(cm)&Ángulo$\theta$\\\hline
			0&25\\\hline
			1&29\\\hline
			2&32\\\hline
			3&37\\\hline
			4&38\\\hline
			5&39\\\hline
			6&40.5\\\hline
			7&43\\\hline
		\end{tabular}
		\hspace{1.5cm}
		\begin{tabular}{|c|c|}
			\hline
			\multicolumn{2}{|c|}{Tercera Medición}
			\\\hline
			Nivel(cm)&Ángulo$\theta$\\\hline
			0&16\\\hline
			1&18\\\hline
			2&22\\\hline
			3&25\\\hline
			4&29\\\hline
			5&32\\\hline
			6&34\\\hline
			7&40\\\hline
		\end{tabular}
		\caption{Datos obtenidos con el prototipo de nivel}
		\label{Tab:DPN}
	\end{table}

Como se puede ver en la Tabla \ref{Tab:DPN}


\end{document}