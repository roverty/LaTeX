\documentclass{article}
\usepackage[spanish]{babel}
\usepackage[utf8]{inputenc}

\title{Tablas en \LaTeX}
\date{}
\begin{document}
\maketitle
Una tabla sencilla:

\begin{table}
\centering
\caption{Una tabla}
\begin{tabular}{|r||c|c|}
	\hline
	1	&	uno		&	one	\\	\hline
	2	&			&	two	\\	
	3	&	tres	&	three\\	\hline \hline
\end{tabular}
\end{table}

\begin{table}
	\begin{tabular}{c|p{300pt}}
		tabla	&	Un texto muy largo que probablemente no quepa en la página.Un texto muy largo que probablemente no quepa en la página.\\ \hline
		figura	&	Otro texto muy largo que probablemente tampoco quepa en la página.Un texto muy largo que probablemente no quepa en la página.\\
	\end{tabular}
\end{table}

Crear una tabla que combine columnas:
\begin{table}[h]
	\begin{tabular}{|c|c|}
		\hline
		\multicolumn{2}{|c|}{\textbf{Elementos}} \\ \hline
		H	&	Hidrógeno	\\	\hline
		O	&	Oxígeno		\\	\hline
		S	&	Azufre		\\	\hline
		Ta	&	Tantalio	\\	\hline
	\end{tabular}
\end{table}

Otra tabla de ejemplo:
\begin{table}[h]
	\begin{tabular}{|c|c|c|c|c|}
		\hline
		\multicolumn{5}{|c|}{\textbf{Título}}		\\ \hline
		\multicolumn{4}{|c|}{Entradas}	&	Salida	\\ \hline
		A	&	B	&	C	&	D		&	S		\\ \hline
		0	&	0	&	0	&	1		&	1		\\ \hline
		0	&	1	&	1	&	0		&	0		\\ \hline
	\end{tabular}
\end{table}

\end{document}