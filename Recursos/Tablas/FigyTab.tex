\documentclass{article}
\usepackage[utf8]{inputenc}
\usepackage[spanish]{babel}
\usepackage{amsmath}
\usepackage{amsfonts}
\usepackage{amssymb}
\usepackage{graphicx} %Paquete para incluir imagenes en un documento LaTeX
\usepackage{vmargin}	%Para definir los margenes del documento
\usepackage{wrapfig}
\usepackage{lipsum}
\usepackage{subcaption}
\usepackage{slashbox}
\usepackage{colortbl}
\usepackage[table]{xcolor}
\usepackage{multicol, multirow}

\setpapersize{A4}	%Define el formato del documento
\setmargins{2.5cm}	%Define el margen izquierdo
{1.5cm}		%Margen superior
{16.5cm} 	%Ancho del area de impresion
{23.42cm}	%Longitud del area de impresion
{0pt}		%Espacio para el encabezado
{5mm}		%Espacio entre el encabezado y el texto
{0pt}		%Espacio para el pie de pagina
{5mm}		%Espacio entre el texto y el pie de pagina

\title{Uso de Imagenes y tablas}
\author{Armando Rivera}
\date{\today}
\definecolor{myGreen}{HTML}{36A736}	%Este codigo es del color verde
\definecolor{myBlue}{HTML}{02528F}
\definecolor{myOrange}{HTML}{FF4312}
\definecolor{blue254}{HTML}{02528F}


\begin{document}
	\maketitle
	\renewcommand{\contentsname}{Tabla de contenido} %Este comando cambia el nombre de "Indice" por "Tabla de contenido"
	\renewcommand{\listfigurename}{Lista de Figuras}
	\renewcommand{\figurename}{Fig.}	%Este comando cambia la palabra figura en las imagenes por Fig.
	\renewcommand{\tablename}{Tabla}
	
	\renewcommand{\listtablename}{Lista de Tablas}
	\tableofcontents	%Coloca un indice
	\listoffigures		%Coloca un indice de figuras
	\listoftables
	
	\section{Objetos Flotantes}
	Un objeto tal como una figura o tabla debe aparecer lo mas proximo al texto que hace referencia a el, sin embargo, al agregar cambios en el documento, los objetos pueden desplazarse de manera inadecuada. \LaTeX resuelve el problema manipulando figuras y tablas como objetos flotantes.
	\\\textcolor{myGreen}{$\backslash$begin}\{figure\}[ubicacion]\\\dots\\\textcolor{myGreen}{$\backslash$end}\{figure\}\\\textcolor{myGreen}{$\backslash$begin}\{table\}[ubicacion]\\\dots\\\textcolor{myGreen}{$\backslash$end}\{table\}\\La \textit{ubicacion} puede ser \textcolor{myGreen}{t}(top), \textcolor{myGreen}{b}(bottom) o \textcolor{myGreen}{h} (here).\\Los comandos \textcolor{myGreen}{$\backslash$tablename} y \textcolor{myGreen}{$\backslash$figurename} redefinidos dentro del documento modifican el encabezado del \textit{caption} de tablas y figuras respectivamente.
	\section{Incluir Imagenes y Figuras}
	El ambiente \textcolor{myGreen}{figure} permite definir un objeto flotante que corresponde a imagenes. Ejemplo:\\
	\begin{figure}[ht]
		\centering
%		\includegraphics[scale=0.45]{/home/armando/Documentos/LaTeX/rdm2.jpeg}
		%Tambien es posible cambar las dimensiones sustituyendo scale por width y height, Ejemplo: [width = 3cm, height = 1cm]
		\caption{Reliquias de a muerte}
	\end{figure}

	Es muy util crear una carpeta donde gurademos todas las imagenes, e incluir el \textit{path} en el preabmbulo.\\\textcolor{myOrange}{$\backslash$graphicspath}\{\{./images/\}\}\\ Despues para seleccionar la figura deseada en \textcolor{myGreen}{$\backslash$includegraphics} no es necesario poner la extencion de la imagen ni toda la ruta, basta con solo el nombre de la Imagen

	\subsection{Paquete wrapfig}
	Es necesario agregar el paquete wrapfig
	El paquete \textcolor{myGreen}{wrapfig} provee los ambientes \textcolor{myGreen}{wrapfigure} y \textcolor{myGreen}{wraptable} que permiten escribir texto alrededor de una figura o tabla.
	\\\\\textcolor{myGreen}{$\backslash$begin}\{wrapfigure\}[P1]\{P2\}[P3]{P4}\\\textcolor{myGreen}{$\backslash$includegraphics}\dots\\\textcolor{myGreen}{$\backslash$end}\{wrapfigure\}\\\\
	Los parametros opcionales \textbf{P1} y \textbf{P3} indican el numero de lineas que ocupara la figura o tabla y el espacio de separacion entre esta y el texto, respectivamente.\\Los parametros obligatorios \textbf{P2} y \textbf{P4} indican la ubicacion (L, R, I, O) y el ancho respectivamente.
	%Le indico que voy a utilizar 13 lineas, ubicacion a la izquierda, 5mm entre el texto y la figura, ancho del area donde se va a ubicar la figura es del 35% del ancho del texto
	\begin{wrapfigure}[13]{l}[5mm]{0.35\textwidth}
	\centering
	\includegraphics[scale=0.45]{/home/armando/Documentos/Cursos/LaTeX/rdm.jpeg}
	\caption{Reliquias}
	\end{wrapfigure}
	\lipsum[1-2] 

	\subsection{El paquete subcaption}
	El paquete \textcolor{myGreen}{$\backslash$subcaption} permite, mediante el ambiente \textcolor{myGreen}{$\backslash$subfigure} utilizar subfiguras cada una con su \textit{caption}.\\\\\textcolor{myGreen}{$\backslash$begin}\{figure\}[ht]\\ \textcolor{myGreen}{$\backslash$centering}\\ \textcolor{myGreen}{$\backslash$begin}\{subfigure\}[posicion]\{ancho\}\\\textcolor{myGreen}{$\backslash$centering}\\\textcolor{myGreen}{$\backslash$includegraphics}\dots\\\textcolor{myGreen}{$\backslash$caption}\{sub caption a.\}\\\textcolor{myGreen}{$\backslash$end}\{subfigure\}\\\textcolor{myGreen}{$\backslash$end}\{subfigure\}\\\textcolor{myGreen}{$\backslash$hfill}\\\textcolor{myGreen}{$\backslash$begin}\{subfigure\}[posicion]\{ancho\}\\\textcolor{myGreen}{$\backslash$centering}\\\textcolor{myGreen}{$\backslash$includegraphics}\dots\\\textcolor{myGreen}{$\backslash$caption}\{sub caption a.\}\\\textcolor{myGreen}{$\backslash$end}\{subfigure\}\\\textcolor{myGreen}{$\backslash$end}\{subfigure\}\\\textcolor{myGreen}{$\backslash$caption}\{General\}\\\textcolor{myGreen}{$\backslash$end}\{figure\}\\
	El paquete \textbf{subcaption} permite ubicar \texttt{subfiguras} cada una con su respectivo \texttt{caption}, dentro de un solo ambiente \textbf{figure}
	\begin{figure}[ht]
		\centering
		\begin{subfigure}[t]{0.475\textwidth}
			\centering
			\includegraphics[scale=0.45]{/home/armando/Documentos/Cursos/LaTeX/rdm.jpeg}
			\caption{Varita, Piedra y capa}
		\end{subfigure}
		\hfill	%Rellena el espacio horizantal que va a quedar entre las dos figuras
		\begin{subfigure}[t]{0.475\textwidth}
			\centering
			\includegraphics[scale=0.55]{/home/armando/Documentos/Cursos/LaTeX/rdm2.jpeg}
			\caption{Reliquias }
		\end{subfigure}
		\caption{Dos Subfiguras de las reliquias de la muerte}
	\end{figure}

	\section{Ambiente Tabular}
	
	El ambiente \textcolor{myGreen}{$\backslash$tabular} permite crear arreglos de datos con o sin bordes\\\\
	\textcolor{myGreen}{$\backslash$begin}\{tabular\}\{\textbar c \textbar c \textbar\}\\\textcolor{myGreen}{$\backslash$hline}\\
	celda 11 \& celda 12\\
	\textcolor{myGreen}{$\backslash$hline}\\
	celda21 \& celda 22\\
	\textcolor{myGreen}{$\backslash$hline}\\
	\textcolor{myGreen}{$\backslash$end}\{tabular\}\\\\
	Los comandos \textcolor{myGreen}{$\backslash$tabcolsep} y \textcolor{myGreen}{$\backslash$arraystretch} modifican el espacio horizontal y vertical entre columnas y filas repectivamente, el primero recibe como parametro de entrada una dimension, el segundo un valor.\\\\
	\textcolor{myGreen}{$\backslash$renewcommand}[$\backslash$tabcolsep]\{dimension\}\\\textcolor{myGreen}{$\backslash$renewcommand}[$\backslash$arraystretch]\{valor\}\\\\
	El comando \textcolor{myGreen}{$\backslash$arrayrulewidth} modifica el gorsor de las lineas de la tabla.
	
	
	\subsection{Construccion de tablas}
	El ambiente \textcolor{myGreen}{table} define un objeto flotante de tipo \textbf{tabla} y el ambiente \textcolor{myGreen}{tabular} define el arreglo en filas y columnas, el uso basico del ambiente \textcolor{myGreen}{tabular} es identico al uso de cualquiera de los ambientes para definicion de matrices.
	
	
	%Separa las columnas, separacion horizontal	
	\renewcommand{\tabcolsep}{10pt}
	
	%Separa las filas, separacion vertical
	\renewcommand{\arraystretch}{1.5}
	
	%modifica el grosor de la linea
	\renewcommand{\arrayrulewidth}{1pt}
	
	
	\begin{wraptable}[11]{L}[5mm]{0.4\textwidth}
		
		\centering
		\begin{tabular}{ccc}	%Tabla de tres columas alineadas al centro
			$\mathbf{x}$ & $\mathbf{y}$ & $\mathbf{f_{xy}(x,y)}$\\
			\hline %Dibuja una linea
			-$1$ & $-2$ & $\frac{1}{8}$\\
			$-0.5$ & $-1$ & $\frac{1}{4}$\\
			$0.5$ & $1$ & $\frac{1}{2}$ \\
			$1$ & $2$ & $\frac{1}{8}$\\
			\hline  
		\end{tabular}
		
		\caption{Tabla con espacios automaticos}
	\end{wraptable}
	\lipsum[1-3]
	Algunas veces dividir una celda en diagonal es algo util, una de las formas de conseguirlo es utilizar el comando \textcolor{myGreen}{backslashbox} del paquete \textcolor{myGreen}{slashbox}, este no es un paquete estandar.
	
	\begin{table}[ht]
		\centering
		\begin{tabular}{|l||c|c|c|}
			\hline
			\backslashbox{Adicion}{Cesion} & CNC & CNS & CM \\
			\hline \hline
			CNC &  & & \\
			\hline
			CNS & & & \\
			\hline
			CM & & & \\
			\hline
		\end{tabular}
		\caption{Tabla en diagonal}
	\end{table}
	
	
	\subsection{Colores en tablas}
	Los comandos \textcolor{myGreen}{$\backslash$rowcolor}, \textcolor{myGreen}{$\backslash$columncolor} y \textcolor{myGreen}{$\backslash$cellcolor} agregan color de fondo a fils, columnas y celdas correspondientemente.\\
	
	El comando \textcolor{myGreen}{$\backslash$rowcolor}\{color\} se incluye justo antes de la fila que se quiere colorear\\
	El comando \textcolor{myGreen}{$\backslash$rowcolors} de la libreria \texttt{table} del paquete \textcolor{myGreen}{xcolor}, puesto justo antes de iniciar la tabla alterna el color entre filas.\\\\
	\textcolor{myGreen}{$\backslash$rowcolors}\{n-fila\}\{color fila\}\{color fila 2\}\\\\
	El comando \textcolor{myGreen}{$\backslash$colorcolumn}\{color\} se incluye al definir la alineacion de las columnas.\\\\\textcolor{myGreen}{$\backslash$begin}\{tabular\}\{\textbar c \textbar $>$\}\{\textcolor{myGreen}{$\backslash$columncolor}\{color\}\}l\textbar r \textbar\}\\\\
	Si se quiere colorear el fondo de una celda especifica se utiliza el comando \textcolor{myGreen}{$\backslash$cellcolor}\{color\}\{texto\} en la ubicacion de la celda.\\\\
	Para agregar color a las tablas en necesario agregar el paquete \textcolor{myGreen}{colortbl} en el preambulo, tambien es necesario agregar el comando \textcolor{myOrange}{$\backslash$usepackage[table]\{xcolor\}} para usar colores en tablas.
	\\\\
	\begin{table}[ht]
		\centering
		\begin{tabular}{|c|c|>{\columncolor{myOrange!50}}c|>{\columncolor{myGreen!50}}c|}
			
			\hline
			\rowcolor{blue254!50}	%El signo significa el porcentaje del color que le estoy indicando
			\textbf{Clase} & $\mathbf{x_i}$ & $\mathbf{f_i}$ & $\mathbf{h_i}$\\\hline
			$[5, 10]$ & 7.5 & 5 & 0.5\\
			\hline 
			$[5, 10]$ & 7.5 & 5 & 0.5\\
			\hline 
			$[5, 10]$ & 7.5 & 5 & 0.5\\
			\hline 
			$[5, 10]$ & \cellcolor{myGreen}{7.5} & 5 & 0.5\\
			\hline 
			$[5, 10]$ & 7.5 & 5 & 0.5\\
			\hline 
		\end{tabular}
	\caption{Tabla con colores}
	\end{table}

		Ejemplo de como alternar colores en tablas.\\
		\begin{table}[ht]
			\centering
			\rowcolors{1}{blue254!50}{myGreen}
			\begin{tabular}{lc}
				\hline
				Poblacion epadronada en Espania & 46.600.949\\
				Poblacion espaniola & 41.882.085\\
				Poblacion Extranjera & 4.718.864(10.1\%)\\
				Poblacion extranjera de 16 anios & 16 \%\\
				Poblacion extranjera $<$ 16 anios & 15.8\%\\
				Paises de procedencia mas frecuentes &\\
				\begin{tabular}{lc}
					Rumania & 15.9\% \\
					Marruecos & 15.8\% \\
					China & 4.05\% \\
				\end{tabular}& Aqui no hay nada	\\
			\hline		
			\end{tabular}
			\caption{Ejemplo de colores alternados en tablas}
		\end{table}

\subsection{Combinar celdas}
Combinar celdas es una tarea que se consigue mediante los paquetes \textcolor{myGreen}{multicol} y \textcolor{myGreen}{multirow}.\\El comando \textcolor{myGreen}{$\backslash$multicolumn} permite combinar celdas adyacentes horizontalmente.
\\\\\textcolor{myGreen}{$\backslash$multicolumn}
\{n\_columnas\}\{alineacion\}\{Contenido\}\\\\

El comando \textcolor{myGreen}{$\backslash$multirow} permite combinar celdas adyacentes verticalmente.\\\\\textcolor{myGreen}{$\backslash$multirow}\{n\_filas\}\{ancho (*)\}\{Contenido\}\\\\Si se han combinado celdas en las cabeceras de fila probablemente se quiera rotar el texto, para esto se tiene el comando \textcolor{myGreen}{$\backslash$rotatebox}\\\\\textcolor{myGreen}{$\backslash$rotatebox}\{origin = c\}\{angulo\}\{Contenido\}\\\\Al combinar celdas se debe utilizar el comando \textcolor{myGreen}{$\backslash$cline}\{i-f\} para trazar lineas horizontales en las celdas adyacentes.


\begin{table}[ht]
	\centering

\begin{tabular}{|>{\cellcolor{myBlue}}c|c|c|c|}
		\hline
		\rowcolor{blue254!75}
		%Indico que voy a combinar 3 columnas, indico que va a tener borde a la izq y va a estar alineado al centro y despues escribo el texto
		
		%El primer & indica que la primera celda esta vacia
		& \multicolumn{3}{c|}{\textcolor{white}{Tolerancia Resistiva ($\pm$)}}\\
		\cline{2-4}
		\rowcolor{blue254!75}
		& \textcolor{white}{20\%} & \textcolor{white}{10\%} & \textcolor{white}{5\%} \\
		& \multirow{4}{*}{100} & \multirow{2}{*}{100} & 100 \\ 
		\cline{4-4}
		& & & 91 \\
		\cline{3-4}
		& & \multirow{2}{*}{82} & 82 \\
		\cline{4-4}
		& & & 75 \\
		%Le indico que cree una linea desde la fila 2 a 4
		\cline{2-4}
		& \multirow{4}{*}{68} & \multirow{2}{*}{68} & 68 \\
		\cline{4-4}
		& & & 62 \\
		\cline{3-4}
		& & \multirow{2}{*}{56} & 56 \\
		\cline{4-4}
		& & & 51 \\
		\cline{2-4}
		& \multirow{4}{*}{47} & \multirow{2}{*}{47} & 47 \\
		\cline{4-4}
		& & & 43 \\
		\cline{3-4}
		& & \multirow{2}{*}{39} & 39 \\
		\cline{4-4}
		& & & 36 \\
		\cline{2-4}
		& \multirow{4}{*}{33} & \multirow{2}{*}{33} & 33 \\
		\cline{4-4}
		& & & 30 \\
		\cline{3-4}
		& & \multirow{2}{*}{27} & 27 \\
		\cline{4-4}
		& & & 24 \\
		\cline{2-4}
		& \multirow{4}{*}{22} & \multirow{2}{*}{22} & 22 \\
		\cline{4-4}
		& & & 20 \\
		\cline{3-4}
		& & \multirow{2}{*}{18} & 18 \\
		\cline{4-4}
		& & & 16 \\
		\cline{2-4}
		& \multirow{4}{*}{15} & \multirow{2}{*}{15} & 15 \\
		\cline{4-4}
		& & & 13 \\
		\cline{3-4}
		& & \multirow{2}{*}{12} & 12 \\
		\cline{4-4}
		& & & 11 \\
		\cline{2-4}
		%Con el -25 indico que cree 25 filas a partir de donde esto hasta arriba
		%Con el comando rotatebox roto el texto 90 grados, puedo cambiar c por l o r 
		\multirow{-25}{*}{\rotatebox[origin=c]{90}{\textcolor{white}{Valores de Resistencia Estándar}}} & 10 & 10 & 10 \\
		\hline
	\end{tabular}
	\caption{Valores estándar para resistencias con diferente nivel de precisión.}
\end{table}

\end{document}