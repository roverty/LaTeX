\documentclass{article}
\usepackage[spanish]{babel}
\usepackage[utf8]{inputenc}
\usepackage{threeparttable}
\title{Tablas en \LaTeX}
\date{}
\begin{document}
\maketitle
Una tabla sencilla:

\begin{table}[h]
\centering
\caption{Una tabla}
\begin{tabular}{|r||c|c|}
	\hline
	1	&	uno		&	one	\\	\hline
	2	&			&	two	\\	
	3	&	tres	&	three\\	\hline \hline
\end{tabular}
\end{table}

Opcion p\{...\} para ingresar parrafos en tablas:
\begin{table}[h]
	\begin{tabular}{c|p{300pt}}
		tabla	&	Un texto muy largo que probablemente no quepa en la página.Un texto muy largo que probablemente no quepa en la página.\\ \hline
		figura	&	Otro texto muy largo que probablemente tampoco quepa en la página.Un texto muy largo que probablemente no quepa en la página.\\
	\end{tabular}
\end{table}

Crear una tabla que combine columnas:
\begin{table}[h]
	\begin{tabular}{|c|c|}
		\hline
		\multicolumn{2}{|c|}{\textbf{Elementos}} \\ \hline
		H	&	Hidrógeno	\\	\hline
		O	&	Oxígeno		\\	\hline
		S	&	Azufre		\\	\hline
		Ta	&	Tantalio	\\	\hline
		Hg	&	Mercurio	\\	\hline
		Mn	&	Manganeso	\\	\hline
	\end{tabular}
\end{table}

\newpage
Otra tabla de ejemplo:
\begin{table}[h]
	\begin{tabular}{|c|c|c|c|c|}
		\hline
		\multicolumn{5}{|c|}{\textbf{Título}}		\\ \hline
		\multicolumn{4}{|c|}{Entradas}	&	Salida	\\ \hline
		A	&	B	&	C	&	D		&	S		\\ \hline
		0	&	0	&	0	&	1		&	1		\\ \hline
		0	&	1	&	1	&	0		&	0		\\ \hline
	\end{tabular}
\end{table}

Tabla con notas utilizando el entorno \verb|threeparttable|:
\begin{table}[h]
	\centering
	\begin{threeparttable}
		\caption{Este es un título}
		\begin{tabular}{|c|c|}
			\hline
			1	&	Esta es una línea\\ \cline{2-2}
			2	&	Esta es una línea con nota al pie \tnote{1}\\ \cline{2-2}
			3	&	Esta es una línea con otra nota al pie\tnote{2}\\ \cline{2-2}
			4	&	Esta es una línea con la misma nota al pie que la línea 2\tnote{2}\\ \cline{2-2}
			5	&	Esta es una línea con otra nota al pie\tnote{3}\\ \hline
		\end{tabular}
		\footnotesize
		\begin{tablenotes}
			\item[1] Pie de nota de 2 y 4
			\item[2] Pie de nota de 3
			\item[3] Pie de nota de 5
		\end{tablenotes}
	\end{threeparttable}
\end{table}

\end{document}