\documentclass[10pt,xcolor={dvipsnames}]{beamer}
\usetheme[
%%% option passed to the outer theme
%    progressstyle=fixedCircCnt,   % fixedCircCnt, movingCircCnt (moving is deault)
  ]{Feather}
% Change the bar colors:
\setbeamercolor{Feather}{fg=NavyBlue!20,bg=NavyBlue}
% Change the color of the structural elements:
\setbeamercolor{structure}{fg=NavyBlue}
% Change the frame title text color:
\setbeamercolor{frametitle}{fg=black!5}
% Change the normal text colors:
\setbeamercolor{normal text}{fg=black!75,bg=gray!5}
%% Change the block title colors
\setbeamercolor{block title}{use=Feather,bg=Feather.fg, fg=black!90} 


% Change the logo in the upper right circle:
%\renewcommand{\logofile}{example-grid-100x100pt} 
% \renewcommand{\logoscale}{0.55}

% Change the background image on the title and final page.
% It stretches to fill the entire frame!
% \renewcommand{\backgroundfile}{example-grid-100x100pt}

%-------------------------------------------------------
% INCLUDE PACKAGES
%-------------------------------------------------------
\usepackage[utf8]{inputenc}			%Ajustar idioma: español
\usepackage[spanish, es-tabla]{babel}
\spanishdecimal{.}
\usepackage[T1]{fontenc}
% \usepackage{helvet}

%% Load different font packages to use different fonts
%% e.g. using Linux Libertine, Linux Biolinum and Inconsolata
% \usepackage{libertine}
% \usepackage{zi4}

%% e.g. using Carlito and Caladea
\usepackage{carlito}
\usepackage{caladea}
\usepackage{zi4}
\usepackage{xcolor}
\usepackage{soulutf8}
%% e.g. using Venturis ADF Serif and Sans
% \usepackage{venturis}

%-------------------------------------------------------
% DEFFINING AND REDEFINING COMMANDS
%-------------------------------------------------------

% colored hyperlinks
\newcommand{\chref}[2]{
  \href{#1}{{\usebeamercolor[bg]{Feather}#2}}
}

%-------------------------------------------------------
% INFORMATION IN THE TITLE PAGE
%-------------------------------------------------------

\title[] 
{ 
      \textbf{Curso Intersemestral de LaTeX}
}

\subtitle[]
{
      \textbf{Introducción a \LaTeX y presentación del curso}
}

\author[Armando Rivera]
{      Armando Rivera \\
		rivera.armando997@gmail.com
}

\institute[]
{%
      Universidad Nacional Autónoma de México\\
      Programa de Tecnología en Cómputo
}

\date{\today}


\begin{document}
\definecolor{c}{rgb}{1,0.5,0}
{\1% % this is the name of the PDF file for the background
\begin{frame}[plain,noframenumbering] % the plain option removes the header from the title page, noframenumbering removes the numbering of this frame only
  \titlepage % call the title page information from above
\end{frame}}


%-------------------------------------------------------
\section{Presentación}
%-------------------------------------------------------
\begin{frame}{Instructores}
\begin{itemize}
	\item Instructor titular:\\Armando Rivera
	\item Instructores Adjuntos:\\Francisco Pablo y Karina Flores
	\item Instructor Auxiliar:\\Raymundo Pérez
\end{itemize}
\end{frame}
\begin{frame}{Consideraciones}
	\begin{itemize}
		\item Se pasará lista a inicio de clase.
		\item Se puede llegar a la hora que sea, siempre y cuando se ingrese al salón de
		manera discreta.
		\item NO consumir alimentos dentro del salón de clases.
	\end{itemize}
\end{frame}
\begin{frame}{Evaluación}
\begin{center}
	Tareas y ejercicios en clase: 100\%
\end{center}
	Para recibir constancia:
	\begin{itemize}
		\item 80\% mínimo de asistencia.
		\item Calificación mínima de 8.
		\item Realizar la encuesta al final del curso.
	\end{itemize}
\end{frame}
\section{Introducción}
\begin{frame}{¿Qué es LaTeX}
	\LaTeX es un sistema de composición de textos (typesetting system) de alta
	calidad que incluye características diseñadas para la producción de documentos
	científicos y técnicos.
	\begin{figure}[h]
		\includegraphics[scale=0.5]{img/LaTeX}
	\end{figure}
\end{frame}
\begin{frame}{TeX y LaTeX}
L A T
EX está basado en TEX, también un sistema
de composición de textos, creado por Donald
Knuth en 1978.
TEX ofrece instrucciones básicas para el formato
(tipo, tamaño de fuente, espaciado, entre otras)
de un documento, L A TEX añade macros que
facilitan esta tarea.
\begin{figure}[h]
	\includegraphics[scale=0.5]{img/Donald}
\end{figure}
\end{frame}
\begin{frame}{Ventajas de \LaTeX}
\begin{itemize}
	\item Permite al usuario enfocarse en el contenido y no en el formato.
	\item Menor necesidad de interacción con una interfaz gráfica.
	\item Multiplataforma.
	\item Retrocompatible.
	\item Separación de palabras (-)
	\item Open Source
\end{itemize}
\end{frame}
\begin{frame}{Entorno}
	Antes de empezar a crear documentos con LaTeX, se requiere instalar el
	siguiente software:
	\begin{itemize}
		\item Compilador: Se encarga de traducir nuestro código a un archivo pdf.
		\begin{itemize}
			\item TexLive (Windows y Linux) 
			\item MikTex (Windows)
			\item MacTex (Mac)
		\end{itemize}
		\item TexStudio (Windows, Linux, Mac)
		\item TexMaker (Windows)
		\item Overleaf (Online)
	\end{itemize}
\end{frame}



{\1
\begin{frame}[plain,noframenumbering]
\finalpage{\huge\textbf{Happy \TeX ting!!}}
\end{frame}}

\end{document}