\documentclass[10pt,xcolor={dvipsnames}]{beamer}
\usetheme[
%%% option passed to the outer theme
%    progressstyle=fixedCircCnt,   % fixedCircCnt, movingCircCnt (moving is deault)
  ]{Feather}
% Change the bar colors:
\setbeamercolor{Feather}{fg=NavyBlue!20,bg=NavyBlue}
% Change the color of the structural elements:
\setbeamercolor{structure}{fg=NavyBlue}
% Change the frame title text color:
\setbeamercolor{frametitle}{fg=black!5}
% Change the normal text colors:
\setbeamercolor{normal text}{fg=black!75,bg=gray!5}
%% Change the block title colors
\setbeamercolor{block title}{use=Feather,bg=Feather.fg, fg=black!90} 


% Change the logo in the upper right circle:
%\renewcommand{\logofile}{example-grid-100x100pt} 
% \renewcommand{\logoscale}{0.55}

% Change the background image on the title and final page.
% It stretches to fill the entire frame!
% \renewcommand{\backgroundfile}{example-grid-100x100pt}

%-------------------------------------------------------
% INCLUDE PACKAGES
%-------------------------------------------------------
\usepackage[utf8]{inputenc}			%Ajustar idioma: español
\usepackage[spanish, es-tabla]{babel}
\spanishdecimal{.}
\usepackage[T1]{fontenc}
% \usepackage{helvet}

%% Load different font packages to use different fonts
%% e.g. using Linux Libertine, Linux Biolinum and Inconsolata
% \usepackage{libertine}
% \usepackage{zi4}

%% e.g. using Carlito and Caladea
\usepackage{carlito}
\usepackage{caladea}
\usepackage{zi4}
\usepackage{xcolor}
\usepackage{soulutf8}
%% e.g. using Venturis ADF Serif and Sans
% \usepackage{venturis}

%-------------------------------------------------------
% DEFFINING AND REDEFINING COMMANDS
%-------------------------------------------------------

% colored hyperlinks
\newcommand{\chref}[2]{
  \href{#1}{{\usebeamercolor[bg]{Feather}#2}}
}

%-------------------------------------------------------
% INFORMATION IN THE TITLE PAGE
%-------------------------------------------------------

\title[] 
{ 
      \textbf{Curso semestral de LaTeX}
}

\subtitle[]
{
      \textbf{Formato y conceptos básicos de un documento}
}

\author[David Alejandro Silva López]
{      David Alejandro Silva López \\
		david.alejandro.proteco@gmail.com
}

\institute[]
{%
      Universidad Nacional Autónoma de México\\
      Facultad de Ingeniería
}

\date{\today}


\begin{document}
\definecolor{c}{rgb}{1,0.5,0}
{\1% % this is the name of the PDF file for the background
\begin{frame}[plain,noframenumbering] % the plain option removes the header from the title page, noframenumbering removes the numbering of this frame only
  \titlepage % call the title page information from above
\end{frame}}


\begin{frame}{Contenido}{}
\tableofcontents
\end{frame}

%-------------------------------------------------------
\section{Estructura básica }
%-------------------------------------------------------
\begin{frame}{Estructura básica}{}
%-------------------------------------------------------
Un archivo de código en \LaTeX  estará conformado por una variedad de comandos que permitirán dar el formato deseado al documento.
Dicho archivo debe tener una estructura mínima para poder ser procesada por el compilador.
\begin{block}{Estructura mínima}
	{\tt\textbackslash documentclass\{...\}\\
		\textbackslash begin\{document\}\\
		<contenido>\\
		\ldots\\
		\textbackslash end\{document\}\}}
\end{block}
\end{frame}

\begin{frame}{Estructura básica}{Preámbulo}
En la sección de código que abarca desde la línea
{\tt \textbackslash documentclass\{...\}}\\
hasta {\tt \textbackslash begin\{document\}}\\
Contiene comandos que definen el tipo de documento, inclusión de paquetes, metadatos, configuración de página y variables o parámetros. 
\end{frame}

\begin{frame}{Estructura básica}{Clases de documentos}
\begin{tabular}{l|c}
	article & Artículos científicos, documentación, notas \\ 
	book &  Libros\\ 
	hitec & Especificaciones técnicas \\ 
	proc & Actas (basadas en article) \\ 
	report & Reportes, tesis, libros pequeños \\ 
	letter & Cartas \\ 
	beamer & Presentaciones de diapositivas (¡Como esta!) \\ 
\end{tabular} 
\end{frame}

\begin{frame}{Estructura básica}{Parámetros del documento}
\begin{tabular}{l|c}
	{\tt\textbackslash author\{...\}} & Nombre del autor \\ 
	{\tt\textbackslash title\{...\}} &  Título\\ 
	{\tt\textbackslash date\{...\}} &  Fecha\\ 
	{\tt\textbackslash usepackage\{p\}} &  Paquete a utilizar\\ 
\end{tabular} 
\end{frame}

\begin{frame}{Estructura básica}{Contenido}

En la sección de código que abarca desde la línea\\
\begin{center}
	{\tt\textbackslash begin\{document\}}\\
\end{center}
hasta\\
\begin{center}
	{\tt\textbackslash end\{document\}}\\
\end{center}

Esta sección contendrá, como su nombre lo dice, el contenido de nuestro documento, así como los comandos necesarios para dar el formato deseado, o bien, importar contenido de otros archivos.
\end{frame}

%-------------------------------------------------------
\section{Formato de texto}
%-------------------------------------------------------
\begin{frame}{Formato de texto}
%-------------------------------------------------------
\begin{tabular}{lll}
	Efecto & Texto corto & Texto largo \\ 
	\hline
	Normal& {\tt\textbackslash textup\{...\}} & {\tt\textbackslash upshape Texto} \\
	\textbf{Negritas}& 	{\tt\textbackslash textbf\{...\}} & {\tt\textbackslash bfseries Texto} \\
	\textit{Itálicas}& {\tt\textbackslash textit\{...\}} & {\tt\textbackslash itshape Texto} \\
	\textsl{Inclinado}& {\tt\textbackslash textsl\{...\}} & {\tt\textbackslash slshape Texto} \\
	\textsc{VERSALES} & {\tt\textbackslash textsc\{...\}} & {\tt\textbackslash scshape Texto} \\
	\textcolor{c}{Color}& {\tt\textbackslash textcolor\{c\}\{...\}} & {\tt\textbackslash color\{c\} Texto} \\
	\underline{Subrayado}& {\tt\textbackslash underline\{...\}} &  \\
	\emph{Énfasis}& {\tt\textbackslash emph\{...\}} &  \\ 	
\end{tabular} \\

\textbf{Nota}

El efecto de los comandos de texto largo permanece hasta el final del entorno en el que se llama o hasta que se llama a otro comando de tamaño.
\end{frame}

\begin{frame}{Formato de texto}{Incluyendo el paquete soul}
\begin{center}
	\begin{tabular}{ll}
		Efecto & Comando \\
		\hline
		\st{Tachado} & {\tt\textbackslash st\{...\}}\\ 
		\hl{Marcado} & {\tt\textbackslash hl\{...\}}\\
		\so{Espaciado} & {\tt\textbackslash so\{...\}}\\
		\ul{Subrayado} & {\tt\textbackslash ul\{...\}}\\   
	\end{tabular}
\end{center}
\textbf{Nota}

En nuestro caso (español), debido a que hacemos uso de acentos, vamos a incluir la paquetería soulutf8.
\end{frame}

\begin{frame}{Formato de texto}{Tamaño de texto}
\begin{center}
	\begin{tabular}{ll}
		Tamaño & Comando \\
		\hline
		\tiny Texto & {\tt\textbackslash tiny Texto}\\ 
		\scriptsize Texto & {\tt\textbackslash scriptsize Texto}\\
		\footnotesize Texto & {\tt\textbackslash footnotesize Texto}\\
		\small Texto & {\tt\textbackslash small Texto}\\
		\normalsize Texto & {\tt\textbackslash normalsize Texto} \\
		\large Texto & {\tt\textbackslash large Texto}\\
		\Large Texto & {\tt\textbackslash Large Texto}\\
		\LARGE Texto & {\tt\textbackslash LARGE Texto}\\
		\huge Texto & {\tt\textbackslash huge Texto}\\
		\Huge Texto & {\tt\textbackslash Huge Texto}\\
	\end{tabular}
\end{center}
\textbf{Nota}

El efecto de estos comandos permanece hasta el final del entorno en el que se llama o hasta que se llama a otro comando de tamaño.
\end{frame}

\section{Secciones, subsecciones y capítulos}
\begin{frame}{Secciones, subsecciones y capítulos}{}
Ayudan a mantener estructurado nuestro documento, facilitan el indexado y la creación de referencias. En la siguiente tabla se muestran todas las estructuras ordenadas por importancia (la más importante al inicio).
\begin{center}
	\begin{tabular}{ll}
		Estructura & Comando \\
		\hline
		Parte & {\tt\textbackslash part\{...\} }\\ 
		Capítulo & {\tt\textbackslash chapter\{...\} }\\
		Sección & {\tt\textbackslash sect\textit{}ion\{...\} }\\
		Subsección & {\tt\textbackslash subsection\{...\} }\\
		Subsubsección & {\tt\textbackslash subsubsection\{...\} }\\
		Párrafo & {\tt\textbackslash paragraph\{...\} }\\
		Subpárrafo & {\tt\textbackslash subparagraph\{...\} }\\
	\end{tabular}
\end{center}
\textbf{Nota}

Las estructuras {\tt\textbackslash part\{\} } y {\tt\textbackslash chapter\{\} }solo se encuentran disponibles en la clase book y report.
\end{frame}

\section{Caracteres especiales}
\begin{frame}{Caracteres especiales}{}
En \LaTeX tenemos 10 caracteres reservados para el uso de comandos o para alg´un uso especial.
\begin{center}
	\begin{tabular}{ll}
		Caracter & Uso \\
		\hline
		{\tt\textbackslash } & Indicador de comando\\ 
		{\{\}} & Delimitadores\\
		{\#} &  Nombrar argumentos de comando\\
		{\&}& Separa columnas de tabla\\
		{\%}& Comentarios \\
		{$\sim$}& Evita separación de palabras\\
		{\$ \textasciicircum \_}& Fórmulas matemáticas\\
	\end{tabular}
\end{center}
\textbf{Nota}
Para poder escribirlos, se le antecede un caracter de escape: {\textbackslash}.\\
Ej: para escribir {\%} se usa \textbackslash{\%} .
\end{frame}

\section{Listas}
\begin{frame}{Listas}{}
Un elemento muy común en un documento suele ser una lista
de elementos.\\
En \LaTeX, la creación de listas es muy sencilla, se tienen los entornos enumerate e itemize, que dan el formato necesario para crear listas numeradas y no enumeradas, respectivamente. Cada elemento de la lista se identifica con el comando \textbackslash item.
\end{frame}

\begin{frame}[fragile]{Listas}{Ejemplos}
\begin{exampleblock}{Ejemplo}
	\begin{verbatim}
		\begin{itemize}
			\item Un elemento de la lista
			\item Otro elemento de la lista
		\end{itemize}
	\end{verbatim}
\end{exampleblock}

\begin{block}{Resultado}
	\begin{itemize}
		\item Un elemento de la lista
		\item Otro elemento de la lista
	\end{itemize}
\end{block}
\end{frame}

\begin{frame}[fragile]{Listas}{Ejemplos}
\begin{exampleblock}{Ejemplo}
	\begin{verbatim}
	\begin{enumerate}
		\item Primer elemento
		\item Segundo elemento
	\end{enumerate}
	\end{verbatim}
\end{exampleblock}

\begin{block}{Resultado}
	\begin{enumerate}
		\item Primer elemento
		\item Segundo elemento
	\end{enumerate}
\end{block}
\end{frame}

\begin{frame}{Secuencias especiales y otros comandos}
	\begin{enumerate}[a)]
		\item {\tt\textbackslash newline} realiza un salto de línea
		\item {\tt\textbackslash newpage} inserta una página nueva
		\item {\tt\textbackslash ver|...|} inserta el texto que está dentro (útil para código)
		\item {\tt\textbackslash LaTeX} inserta \LaTeX
		\item {\tt\textbackslash TeX} inserta \TeX 
		\item {\tt\textbackslash today} inserta la fecha de hoy
	\end{enumerate}
\end{frame}

{\1
\begin{frame}[plain,noframenumbering]
\finalpage{Una vez conociendo la teoría, estamos listos. ¡Manos a la obra!}
\end{frame}}

\end{document}