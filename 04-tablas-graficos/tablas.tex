%------------------   PREAMBULO  -----------------
\documentclass{article}
\usepackage[utf8]{inputenc}
\usepackage[spanish,es-tabla]{babel} %es-table cambia "cuadro" por "tabla"
\usepackage[top=2cm, bottom=2cm, left=2.5cm, right=2.5cm]{geometry}
\usepackage{multirow, array} % para las tabla
\usepackage[dvipsnames,table]{xcolor} %más colores, "table" para poder ocupar algunos comandos 
\usepackage{colortbl} %para poder usar colores en tablas
\usepackage{longtable} %tablas largas
\usepackage{booktabs} %Intrucciones extras para las tablas
\definecolor{lightgray}{gray}{0.9}

%-------------------- DATOS ---------------------------
\title{TABLAS}
\author{Ana Laura Reynoso }
\date{\today} %Pone la fecha del día de hoy 

%-------------- CONTENIDO DEL DOCUMENTO (CUERPO) ----------
\begin{document}

\maketitle %Habilita la visualización de titulo, autor, fecha, etc.

%------------------------------------- Tabular ----------------------------
\section{El entorno tabular}
Del entorno de trabajo se entra y se sale mediante los comandos \textbf{\textbackslash begin\{tabular\}\{ \dots \}}  y \textbf{\textbackslash end\{tabular\}} respectivamente. 

\begin{itemize}
	\item Para indicar que se ha terminado un renglón se colocan \textbf{\textbackslash 						\textbackslash} al final de dicho renglón.
	\item Para indicar la separación del contenido de cada columna se indica con \& .
	\item c: el contenido de la columna está centrado.
	\item l: el contenido está alineado a la izquierda.
	\item r: el contenido está alineado a la derecha.
	\item Para colocar lineas verticales agregamos $\mid$ 
	\item Para colocar lineas horizontales agregamos \textbackslash hline (una línea tan larga 					como la tabla) o \textbackslash cline$\lbrace$a-b$\rbrace$ donde \textit{a} es el número de columna en la cual inicia el borde y \textit{b} en la cual termina.
	
\end{itemize}

\subsection{Ejemplo 1}
\begin{tabular}{c c c c}
	a & b & c & d \\
	e & f & g & h \\
	i & j & k & l
\end{tabular}

\subsection{Ejemplo 2}
\begin{tabular}{c l r}
	Hola 		  & 5869745.25456 & Ana Laura \\
	Tengo un gato & Azul 		  & 3.2       \\
\end{tabular}


\subsection{Ejemplo 3}
\begin{center} %Centramos esta tabla
	\begin{tabular}{|c|l|r|||} \hline
		Hola 		  & 5869745.25456   & Ana Laura \\ \hline
		Tengo un gato & Azul 		    & 3.2       \\ \hline \hline
		Agua 		  & tengo otro gato & 10000     \\ \hline 
	\end{tabular}
\end{center}

\subsection{Ejemplo 4}
\begin{tabular}{|l|l|l|l|} \hline
	\textbf{Nombre} & Segundo nombre & Apellido paterno & Apellido materno \\ \hline
	Ana	   & Laura          & Reynoso          & Gálvez			  \\ \cline{2-3}
	Pepe   &                & Ramirez		   & Del Toro         \\ \hline
\end{tabular}




%------------------------------- Table --------------------------------
\newpage
\section{El entorno table}
Del entorno de trabajo se entra y se sale mediante los comandos \textbf{\textbackslash begin\{table\} [\dots ]}  y \textbf{\textbackslash end\{table\}} respectivamente. 

Podemos agregar un título y una etiqueta con el entorno \textit{table}. Dentro de este entorno debe de haber un entorno \textit{tabular}.

\begin{itemize}
	\item h: Coloca la tabla en el lugar donde se encuentra el código.
	\item b: Coloca la tabla al final de la página.
	\item t: Coloca la tabla al inicio de la página. 
	\item Se pueden crear tablas con celdas que combinan múltiples columnas con el comando 					  \textit{multicolumn}. Sintaxis: \textbackslash multicolumn $\lbrace$cuántas columnas$\rbrace$ $\lbrace$posición del texto$\rbrace$ $\lbrace$texto$\rbrace$.
	\item Se pueden crear tablas con celdas que combinen múltiples renglones con el comando 				  \textit{multirow}. Sintaxis: \textbackslash multirow $\lbrace$cuántos renglones$\rbrace				$ $\lbrace$ancho de columna$\rbrace$ $\lbrace$texto$\rbrace$. Para el ancho de 						columna podemos poner un asterisco (*) para determinar el tamaño automáticamente.
\end{itemize}

\subsection{Ejemplo 1} 

\begin{table}[h]
	%\centering
	\caption{Tabla de datos} %Titulo de la tabla
	\begin{tabular}{|r|r|r|} \hline 
		Nombre   & Color favorito   & Signo zodiacal \\ \hline
		Laura    & Verde		    & Aries 		 \\
		Fernando & Morado           & Virgo		     \\ \hline
	\end{tabular}
	\label{tab:resultados} %Etiqueda de la tabla
\end{table}

\subsection{Ejemplo 2}
\begin{table}[h]
	\centering 
	\begin{tabular}{|c|c|c|} \hline
		\textbf{Artículo} & \textbf{Contenido neto} & \textbf{Precio} \\ \hline
		Tortillas 		  & 1 kg					& \$ 15			  \\ \hline
		Leche			  & 1 L						& \$ 17			  \\ \hline
		\multicolumn{2}{|c|}{Total}                 & \$29			  \\ \hline
	\end{tabular}
	\caption{Celdas multicolumna}
\end{table}

\subsection{Ejemplo 3}
\begin{table}[h]
	\centering
	\begin{tabular}{|c|c|c|} \hline
		\multirow{4}{*}{Elementos químicos} & Litio & Li \\ \cline{2-3}
		& Sodio & Na \\ \cline{2-3}
		& Potacio & K \\ \cline{2-3}
		& Oro 	  & Au \\ \hline
	\end{tabular}
	\caption{Celdas multirenglón}
\end{table}


%---------------------- Tablas de más de una página -----------------------
\newpage
\section{Tablas de más de una página}

Se debe usar el ambiente \textbf{longtable}, éste se encarga de dividir la tabla entre páginas,
para esto se debe agregar el paquete longtable en el preámbulo:\textbf{\textbackslash usepackage $\lbrace$longtable$\rbrace$} 

\subsection{Ejemplo 1}
\begin{longtable}{c c c} %toprule
	Nombre & Apellido & Edad \\
	Ana & Laura & 22 \\
	Nombre & Apellido & Edad \\
	Ana & Laura & 22 \\
	Nombre & Apellido & Edad \\
	Ana & Laura & 22 \\
	Nombre & Apellido & Edad \\
	Ana & Laura & 22 \\
	Nombre & Apellido & Edad \\
	Ana & Laura & 22 \\
	Nombre & Apellido & Edad \\
	Ana & Laura & 22 \\
	Nombre & Apellido & Edad \\
	Ana & Laura & 22 \\	
	Nombre & Apellido & Edad \\
	Ana & Laura & 22 \\
	Nombre & Apellido & Edad \\
	Ana & Laura & 22 \\
	Nombre & Apellido & Edad \\
	Ana & Laura & 22 \\
	Nombre & Apellido & Edad \\
	Ana & Laura & 22 \\
	Nombre & Apellido & Edad \\
	Ana & Laura & 22 \\
	Nombre & Apellido & Edad \\
	Ana & Laura & 22 \\	Nombre & Apellido & Edad \\
	Ana & Laura & 22 \\	
	Nombre & Apellido & Edad \\
	Ana & Laura & 22 \\
	Nombre & Apellido & Edad \\
	Ana & Laura & 22 \\
	Nombre & Apellido & Edad \\
	Ana & Laura & 22 \\
	Nombre & Apellido & Edad \\
	Ana & Laura & 22 \\	
	Nombre & Apellido & Edad \\
	Ana & Laura & 22 \\
	Nombre & Apellido & Edad \\
	Ana & Laura & 22 \\
	Nombre & Apellido & Edad \\
	Ana & Laura & 22 \\
	Nombre & Apellido & Edad \\
	Ana & Laura & 22 \\
	Nombre & Apellido & Edad \\
	Ana & Laura & 22 \\
	Nombre & Apellido & Edad \\
	Nombre & Apellido & Edad \\
	Ana & Laura & 22 \\
	Nombre & Apellido & Edad \\
	Ana & Laura & 22 \\
	Nombre & Apellido & Edad \\
	Ana & Laura & 22 \\	
	Nombre & Apellido & Edad \\
	Ana & Laura & 22 \\
	Nombre & Apellido & Edad \\
	Ana & Laura & 22 \\
	Nombre & Apellido & Edad \\
	Ana & Laura & 22 \\
	Nombre & Apellido & Edad \\
	Ana & Laura & 22 \\
	Nombre & Apellido & Edad \\
	Ana & Laura & 22 \\
	Nombre & Apellido & Edad \\
	Nombre & Apellido & Edad \\
	Ana & Laura & 22 \\
	Nombre & Apellido & Edad \\
	Ana & Laura & 22 \\
	Nombre & Apellido & Edad \\
	Ana & Laura & 22 \\	
	Nombre & Apellido & Edad \\
	Ana & Laura & 22 \\
	Nombre & Apellido & Edad \\
	Ana & Laura & 22 \\
	Nombre & Apellido & Edad \\
	Ana & Laura & 22 \\
	Nombre & Apellido & Edad \\
	Ana & Laura & 22 \\
	Nombre & Apellido & Edad \\
	Ana & Laura & 22 \\
	Nombre & Apellido & Edad \\
	Nombre & Apellido & Edad \\
	Ana & Laura & 22 \\
	Nombre & Apellido & Edad \\
	Ana & Laura & 22 \\
	Nombre & Apellido & Edad \\
	Ana & Laura & 22 \\	
	Nombre & Apellido & Edad \\
	Ana & Laura & 22 \\
	Nombre & Apellido & Edad \\
	Ana & Laura & 22 \\
	Nombre & Apellido & Edad \\
	Ana & Laura & 22 \\
	Nombre & Apellido & Edad \\
	Ana & Laura & 22 \\
	Nombre & Apellido & Edad \\
	Ana & Laura & 22 \\
	Nombre & Apellido & Edad \\
\end{longtable}


%------------------------------ Colores en tablas -------------------------
\newpage
\section{Colores en tablas}
En el caso de querer colorear una celda, se utiliza el comando \textbf{\textbackslash cellcolor} antes del contenido de la celda.

Recodemos importar el paquete $\lbrace$colortbl$\rbrace$ para poder usar \textbackslash cellcolor. Importar el paquete $\lbrace$xcolor$\rbrace$ para más rangos de colores.

\subsection{Ejemplo 1}
\begin{table}[h]
	\centering
	\begin{tabular}{|l|l|l|} \hline
		Gato & Perro & Delfín \\ \hline
		Pájaro & León & Gusano \\ \hline
		\cellcolor{green} Gallina & Mariposa & Ñandú \\ \hline
	\end{tabular}
	\caption{Tabla con colores}
\end{table}

\subsection{Ejemplo 2}
\begin{table}[h]
	\centering
	\rowcolors{1}{}{lightgray}	% row= renglón
	\begin{tabular}{ll} 
		\rowcolor{Salmon} $x_{n+1}$ & $|x_{n+1}-x_n|$\\ \hline
		1.20499955540054 & 0.295000445\\
		1.17678931926590 & 0.028210236\\
		1.17650193990183 & 3.004$\times10^{-8}$\\
		1.17650193990183 & 4.440$\times10^{-16}$\\ \hline
	\end{tabular}
	\caption{Iteración de números}
\end{table}

\subsection{Ejemplo 3}
\newcolumntype{g}{>{\columncolor{lightgray}}c}
\begin{table}[h]
	\centering
	\begin{tabular}{g l} 
		\rowcolor{Salmon} $x_{n+1}$ & $|x_{n+1}-x_n|$\\ \hline
		1.20499955540054 & 0.295000445\\
		1.17678931926590 & 0.028210236\\
		1.17650193990183 & 3.004$\times10^{-8}$\\
		1.17650193990183 & 4.440$\times10^{-16}$\\ \hline
	\end{tabular}
	\caption{Iteración de números}
\end{table}



\end{document}