\documentclass{article}
\usepackage[utf8]{inputenc}           % Use uft8 enconding to have a wide
                                      % number of symbols available
\usepackage[spanish]{babel}           % Configure the text language
                                      % to know where to put the hyphen in a 
                                      % line break
\usepackage{graphicx}                 % To include images
\usepackage{anysize}                  % Allows marginsize command
\usepackage{fancyhdr}                 % Configure the header and footer  
\usepackage{titlesec}                 % Changes the section titles properties
\usepackage{amsmath}                  % Active wide number of math symbols
\usepackage{amssymb}                  % Math symbols such as semijoin
\usepackage{longtable}                % Multiple-page table
\usepackage[export]{adjustbox}        % Allows to resize tables
\usepackage{enumitem}                 % Controls the item position 
\usepackage{listings}                 % Package for code fences
%\usepackage{xcolor}                   % Create colors
\usepackage[
  table,
  svgnames,
  dvipsnames
]{xcolor}                             % Colors for code fences and 
                                      %table (rowcolor)
\usepackage{textcomp}                 % Helps to display quotes symbols properly
\usepackage{array}                    % Align fix size columns in tables

\decimalpoint%                        % Use dot instead of comma to write 
                                      % decimal numbers
\setlength{\parindent}{0in}           % No indentation at first paragraph

\renewcommand{\familydefault}{\sfdefault} % Changing font
\titleformat*{\section}{\large\bfseries}  % Change section size
\marginsize{1.5cm}{2cm}{1.2cm}{1cm}       % {left}{right}{above}{below}
\setlength{\headsep}{0.3in}               % Changing headsep length
                                          % headsep is the vertical length 
                                          % between header an text area
\lstset{upquote=true}                     % For display quotes and double 
                                          % quoutes in a better style

% Defining column content alignment for fix size columns
\newcolumntype{L}[1]{>{\raggedright\let\newline\\\arraybackslash\hspace{0pt}}m{#1}}
\newcolumntype{C}[1]{>{\centering\let\newline\\\arraybackslash\hspace{0pt}}m{#1}}
\newcolumntype{R}[1]{>{\raggedleft\let\newline\\\arraybackslash\hspace{0pt}}m{#1}}  

%%%%%%%%%%%%%%%%%%%%%%%%%%%%%%%%%%%%%%%%%%%%%%%%%%%%%%%%%%%%%%%%%%%%%%%%%%%%%%%
%%%%%%%%%%                        Code style                         %%%%%%%%%%
%%%%%%%%%%%%%%%%%%%%%%%%%%%%%%%%%%%%%%%%%%%%%%%%%%%%%%%%%%%%%%%%%%%%%%%%%%%%%%%

\definecolor{codegreen}{rgb}{0,0.6,0}
\definecolor{codegray}{rgb}{0.5,0.5,0.5}
\definecolor{codepurple}{rgb}{0.58,0,0.82}
\definecolor{backcolour}{rgb}{1,1,1}

\lstdefinestyle{mystyle}{
  backgroundcolor=\color{backcolour},   
  commentstyle=\color{codegreen},
  keywordstyle=\color{magenta},
  numberstyle=\tiny\color{codegray},
  stringstyle=\color{codepurple},
  %
  basicstyle=\ttfamily\footnotesize,
  captionpos=b,                    
  breakatwhitespace=false,         
  breaklines=true,                 
  keepspaces=true,                 
  showspaces=false,                
  showstringspaces=false,
  showtabs=false,                  
  %
  tabsize=2
  % Diplay number to the left
  % numbers=left,                    
  % numbersep=5pt,                  
}

\lstset{style=mystyle}


%%%%%%%%%%%%%%%%%%%%%%%%%%%%%%%%%%%%%%%%%%%%%%%%%%%%%%%%%%%%%%%%%%%%%%%%%%%%%%%
%%%%%%%%%%                        Header Style                       %%%%%%%%%%
%%%%%%%%%%%%%%%%%%%%%%%%%%%%%%%%%%%%%%%%%%%%%%%%%%%%%%%%%%%%%%%%%%%%%%%%%%%%%%%

\pagestyle{fancy}
\fancyhf{}
\renewcommand{\headrulewidth}{0pt}

% Right header
% The right header has 
%   the subject title,
%   subtitle and 
%   the university logo
\fancyhead[R]{
    \begin{tabular}{l}
        \materia \\ 
        \actividad%
    \end{tabular}
    \,% Adding space between titles and logo    
    \rule[-1.75\baselineskip]{0pt}{0pt}
    % Strut to ensure a 1/4 \baselineskip between image and header rule
    \includegraphics[height=3\baselineskip,valign=c]{unam-negro}
}

%%%%%%%%%%%%%%%%%%%%%%%%%%%%%%%%%%%%%%%%%%%%%%%%%%%%%%%%%%%%%%%%%%%%%%%%%%%%%%%
%%%%%%%%%%              Cover page generator command                 %%%%%%%%%%
%%%%%%%%%%%%%%%%%%%%%%%%%%%%%%%%%%%%%%%%%%%%%%%%%%%%%%%%%%%%%%%%%%%%%%%%%%%%%%%

\newcommand{\coverPage}{
\thispagestyle{empty}
  \begin{minipage}[t][5cm][t]{0.2\linewidth}
    \includegraphics[width=2.5cm]{unam-negro}

    \vspace{10cm}
    % The following space is mandatory to display correct layout

    \includegraphics[width=2.5cm]{fi-negro}
  \end{minipage}
  %
  \begin{minipage}[t]{0.7\linewidth}
    \vspace{-2.5cm}
    \LARGE{\textbf{\university}}\\
    \Large{\textbf{\faculty}} \\
  
    \large{\semestre}\\[2cm]
  
    \large{\textbf{\materia (\clave)}}\\
    \large{\textbf{Gpo: \grupo}}\\[5mm]
    \large{\textbf{Profesor:} \profesor}\\ [1.5cm]
    \begin{center}
        \LARGE{\textbf{\actividad}}\\
        \LARGE{\textbf{\titulo}}\\
    \end{center}
  
    \vspace{3.3cm}
  
    %\large{\textbf{Alumno:} \alumno} \\[1.5cm]
    \large{
      \begin{itemize}[ noitemsep, align=left ]
        \item [\textbf{Alumno(s):}] 
          \begin{flushright}
            \alumno
          \end{flushright}
      \end{itemize}
    } \vspace{1.5cm}
  
    \begin{flushright}
        \fechaEntrega%
    \end{flushright}
  \end{minipage}

\newpage
}

\begin{document}

%%%%%%%%%%%%%%%%%%%%%%%%%%%%%%%%%%%%%%%%%%%%%%%%%%%%%%%%%%%%%%%%%%%%%%%%%%%%%%%
%%%%%%%%%%                Variables definition                       %%%%%%%%%%
%%%%%%%%%%%%%%%%%%%%%%%%%%%%%%%%%%%%%%%%%%%%%%%%%%%%%%%%%%%%%%%%%%%%%%%%%%%%%%%

\newcommand{\university}{Universidad Nacional Autónoma de México}
\newcommand{\faculty}{Facultad de Ingeniería}
\newcommand{\semestre}{2021-1}
\newcommand{\materia}{BDD}
\newcommand{\clave}{294}
\newcommand{\grupo}{1}
\newcommand{\profesor}{Ing. Rodriguez Campos \textsc{Jorge Alberto}}

%\newcommand{\alumno}{Francisco Pablo \textsc{Rodrigo}}
\newcommand{\alumno}{
  Francisco Pablo \textsc{Rodrigo} \\ 
  Flores Martinez \textsc{Emanuel}   
}
\newcommand{\actividad}{Proyecto Final}
\newcommand{\titulo}{BDD empleada para administrar ventas de autos}

\newcommand{\fechaEntrega}{}

\newcommand{\codedir}{codigo}
\graphicspath{ {latex/assets/}{bdd_proyecto.assets/}{modelo} }

\coverPage%


%%%%%%%%%%%%%%%%%%%%%%%%%%%%%%%%%%%%%%%%%%%%%%%%%%%%%%%%%%%%%%%%%%%%%%%%%%%%%%%
%%%%%%%%%%                        Contents                           %%%%%%%%%%
%%%%%%%%%%%%%%%%%%%%%%%%%%%%%%%%%%%%%%%%%%%%%%%%%%%%%%%%%%%%%%%%%%%%%%%%%%%%%%%

\section*{Tabla de asignación de sitios}

\newcounter{nodeCounter}
\newcommand{\nameTabBuilder}[1]{F\_RFP\_#1}
\newcommand{\snI}{RFP\_S1}
\newcommand{\snII}{RFP\_S2}
\newcommand{\snIII}{EFM\_S1}
\newcommand{\snIV}{EFM\_S2}
\newcommand{\pdbI}{rfpbd\_s1.fi.unam}
\newcommand{\pdbII}{rfpbd\_s2.fi.unam}
\newcommand{\pdbIII}{efmbd\_s1.fi.unam}
\newcommand{\pdbIV}{efmbd\_s2.fi.unam}

{
  \setlength\tabcolsep{3.5mm}
  \def\arraystretch{2}          % Do not define globally (for that reason we
                                % enclose table inside brackets)
  \begin{longtable}{
    |C{0.05\linewidth}
    |p{0.5\linewidth}
    |C{0.2\linewidth}
    |p{0.1\linewidth}|}
  \hline
  %%%%% Start: Table header 
  \textbf{Núm. nodo} &
  \textbf{Características} & 
  \textbf{Nombre global del PDB} & 
  \parbox[t]{2cm}{\centering \textbf{Sufijo para fragmentos}} 
  \\ \hline
  %%%%% End: Table header 
  %
  % row 1
  \stepcounter{nodeCounter} \arabic{nodeCounter} &
  Se encuentra en la región AME y es el servidor con la mayor capacidad 
  de procesamiento & 
 \pdbI & 
  \snI 
  \\ \hline
  %
  % row 2
  \stepcounter{nodeCounter} \arabic{nodeCounter} &
  Se encuentra en la región EUR & 
  \pdbII & 
  \snII
  \\ \hline
  %
  % row 3
  \stepcounter{nodeCounter} \arabic{nodeCounter} &
  \begin{flushleft} 
  Cuenta con una VPN que conecta al servidor con las oficinas de los dueños de 
    la empresa, así como herramientas para cifrado de datos. \\[3mm]
  Se encuentra en las oficinas centrales de la empresa en USA. \end{flushleft} & 
  \pdbIII & 
  \snIII
  \\ \hline
  %
  % row 4
  \stepcounter{nodeCounter} \arabic{nodeCounter} &
  Cuenta con herramientas para realizar procesamiento de contenido multimedia. 
  Así como una gran capacidad de almacenamiento. Se encuentra en las oficinas 
  centrales de la empresa en USA. & 
  \pdbIV & 
  \snIV
  \\ \hline
  \end{longtable}
}



\end{document}
